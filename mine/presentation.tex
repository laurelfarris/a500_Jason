\documentclass{beamer}
\usepackage{verbatim}
\setlength{\parindent}{0em}
\setlength{\parskip}{0.5em}
\usepackage{graphicx}
%\usepackage{mdwlist}
\usepackage[mathscr]{euscript}
%\usepackage{amssymb}
\setbeamertemplate{itemize item}[circle]
\setbeamertemplate{itemize subitem}[square]
\definecolor{mine}{RGB}{155,155,155}
\definecolor{fpink}{rgb}{0.99, 0.56, 0.67}
\definecolor{oj}{rgb}{1.0,0.65,0.0}
\definecolor{cblue}{rgb}{0.39, 0.58, 0.93}
\definecolor{turq}{rgb}{0.0, 0.81, 0.82}
\definecolor{cgreen}{rgb}{0.0, 0.8, 0.6}
\definecolor{cpink}{rgb}{0.92, 0.3, 0.26}
\setlength{\unitlength}{1mm}

\setbeamercolor{normal text}{bg=black, fg=white}
\setbeamercolor{title}{fg=turq}
\setbeamercolor{frametitle}{fg=cgreen}
\setbeamercolor{framesubtitle}{fg=cblue}
%\setbeamercolor{block title}{fg=green}
\setbeamercolor{itemize item}{fg=fpink}
\setbeamercolor{itemize subitem}{fg=cpink}
\setbeamercolor{enumerate item}{fg=fpink}

\usefonttheme{serif}
\setbeamerfont{framesubtitle}{size=\large}
\setbeamerfont{frametitle}{series=\bfseries}
%\usepackage{fontspec}
%\setmainfont{}

\begin{document}

\begin{frame}
    \begin{centering}
    {\Large Basic Principles of Solar Acoustic Holography}
    {\large\textcolor{cblue}{Laurel Farris}}\\
    {\large\textcolor{cblue}{ASTR 500}}\\
    {\textcolor{cblue}{11 March 2016}}\\
    \end{centering}
    \vspace{1cm}
    \includegraphics[width=\paperwidth]{starwars.jpg}
\end{frame}

\begin{comment}
\begin{frame}{Outline}
    \begin{enumerate}
        \item Basic principles of helioseismic holography
        \item Historical stuff
        \item Simple acoustic power holography
        \item Phase-sensitive holography
        \item Examples of applications
    \end{enumerate}
\end{frame}
\end{comment}

\begin{frame}{The Basic Principle of Heliospheric Holography}
    Defined as the \emph{phase-coherent} computational
    reconstruction of the acoustic field in the solar interior
    or far side of the sun.
\end{frame}

\begin{frame}{Comparison with optics}
    \begin{itemize}
        \item Submerged sources in sun $\sim$ things we see
        \item Photosphere $\sim$ surface of cornea (front of eye)
    \end{itemize}
    Both involve refocusing radiation to render stigmatic images
    that can be sampled over focal surface at any desired depth.
    \begin{figure}
        \includegraphics[width=\textwidth]{eye.png}
    \end{figure}
\end{frame}

\begin{frame}{Historical Note}
    \begin{itemize}
        \item Concept proposed in 1975 by Roddier
        \item Developed over the 1990s by Lindsey and Braun\\
            $\rightarrow$ Key to locating and examining
            fine structure in the interior and far side of the sun.
    \end{itemize}
    Seismic holography was first applied to helioseismic data from
    the SOlar Heliospheric Observatory (SOHO).\\
    ``New'' (1998 - 1999) solar acoustic phenomena:
    \begin{itemize}
        \item `acoustic moats' surrounding sunspots
        \item `acoustic condensations' 10\-20 Mm beneath active regions
        \item `acoustic glories' surrounding complex active regions
        \item first helioseismic images of a flare
    \end{itemize}
\end{frame}
%----------------------------------------------------------------%
\begin{frame}{Motivation}
    Space weather application $\rightarrow$
    See flares before they turn to face Earth!
\end{frame}

%----------------------------------------------------------------%
\begin{frame}{``Basic Principles of Solar Acoustic Holography''}
    {C. Lindsey and D. C. Braun; 2000}
    \begin{itemize}
        \item Compare simple acoustic-power to phase-sensitive
        \item Propose ``simple computational principles'' to produce images
            from high quality helioseismic observations.
    \end{itemize}
\end{frame}

\begin{frame}{Step 1: Helioseismic observations}
    \begin{figure}
        \includegraphics[width=\textwidth]{fig_1.pdf}
    \end{figure}
    \begin{itemize}
        \item Interior sources of acoustic waves
        \item Pattern of ripples on surface
            directly \emph{above} the sources.
        \item The waves are \emph{absorbed} upon reaching the surface,
            not reflected back down or propagated into atmosphere.
            (accurate for $\nu >\ \sim 5.5$ mHz)
    \end{itemize}
\end{frame}

\begin{frame}{Step 2: Apply observations to a model}
    \begin{figure}
        \includegraphics[width=\textwidth]{fig_2.png}
    \end{figure}
    Disturbance now propagating back down.\\
    Consider a ``focal plane'' in the interior, at a depth
    $z_{\textrm{plane}}$.
    \begin{itemize}
        \item $z_{\textrm{plane}} = z_{\textrm{source}}$:
            diffraction-limited signature
        \item $z_{\textrm{plane}} \neq z_{\textrm{source}}$:
            unfocused, diffuse profile
    \end{itemize}
\end{frame}

\begin{frame}{Absorbers confined to infinitely thin sheets}
    \begin{figure}
        \includegraphics[width=0.9\textwidth]{fig_3.png}
    \end{figure}
    \vspace{-0.5cm}
    \begin{columns}
        \column{0.5\paperwidth}
            \begin{itemize}
                \item $z_{\textrm{SOHO}} = 0$ Mm
                \item $z_{\textrm{MDI}} = 56$ Mm
                    ($\sim \frac{1}{10}$ R$_{\odot}$).
            \end{itemize}
        \column{0.5\paperwidth}
            \begin{itemize}
                \item (b) \& (c) acoustic \emph{stalactites}
                \item (d) \& (e) acoustic \emph{stalagmites}
            \end{itemize}
    \end{columns}
\end{frame}

%-------------------------------------------------------------------%
\begin{frame}{The Computational Task}
    Two perspectives:
    \begin{enumerate}
        \item ``spectral''
            \begin{itemize}
                \item wavenumber-frequency $(k,\nu)$
                \item (computationally advantageous)
            \end{itemize}
        \item ``time distance''
            \begin{itemize}
                \item space-time $(x,t)$
                \item (more intuitive)
            \end{itemize}
    \end{enumerate}
    Given
    \begin{itemize}
        \item acoustic amplitude
        \item derivative normal to surface surrounding medium that is
            free of sources, sinks, or scatterers
    \end{itemize}
    can extrapolate the (incomplete) acoustic field in the interior.
\end{frame}

\begin{frame}{The space-time perspective}{Acoustic \emph{egression}}
    \begin{itemize}
        \item $\psi(\mathbf{r}',t')$: \emph{Actual} acoustic field
        \item $H$: \emph{incomplete regression} of the acoustic field
            \begin{itemize}
                \item $H_{+}(\mathbf{r},z,t)$ ``acoustic egression''\\
                    ``focal point'' $(\mathbf{r},z,t)$
                    $\rightarrow$ surface $(\mathbf{r}',0,t')$.
                \item $H_{-}$ ``acoustic ingression'' (next slide)
            \end{itemize}
    \end{itemize}
    $$ H_{+}(\mathbf{r},z,t) = \int
    \textrm{d}t' \int\limits_{a<|\mathbf{r}-\mathbf{r}'|<b}
    \textrm{d}^{2}r'G_{+}
    (|\mathbf{r}-\mathbf{r}'|,z,t-t')\psi(\mathbf{r}',t')  $$
    \textcolor{yellow}{$G_{+}$} $\rightarrow$
    Green's function that expresses how a single transient point
    disturbance propagates forward or backward in time between
    $(\mathbf{r}',0,t')$ and $(\mathbf{r},z,t)$.

    After computing $H_{+}$, square and integrate it to produce an egression
    power map over the time period in desired range.
\end{frame}

\begin{frame}{The space-time perspective}{Acoustic \emph{ingression}}
    $H_{-}$: ``acoustic ingression'', the time reversal
    of $H_{+}$; waves converge into the focal point
    and contribute to the disturbance.

    Green's function:
    $$ G_{-}(|\mathbf{r}-\mathbf{r'}|,z,t-t') =
    G_{+}(|\mathbf{r}-\mathbf{r'}|,z,t'-t) $$
\end{frame}

\begin{comment}
\begin{frame}
    \begin{picture}(60,40)
        % first circle
        \put(20,20){\circle{20}}
        \put(20,20){\circle{10}}
        \put(20,20){\vector(3,0){7}}
        \put(20,20){\vector(0,5){10}}
        % second circle
        \put(60,20){\circle{20}}
    \end{picture}
\end{frame}
\end{comment}
%-------------------------------------------------------------------%
\begin{frame}{The wavenumber-frequency perspective}
    \begin{itemize}
        \item $\hat{\psi}(\mathbf{k},\nu)$: Fourier transform of $\psi(\mathbf{r},t)$
        \item $\hat{G}_{+}(|\mathbf{k}|,z,\nu)$: Fourier transform of
            $G_{+}(|\mathbf{r}|,z,t)$
    \end{itemize}
    Fourier transform of the egression:
    $$ \hat{H}_{+}(\mathbf{k},z,\nu) = \hat{G}_{+}(|\mathbf{k}|,z,\nu)
     \hat{\psi}(\mathbf{k},\nu) $$
    from the convolution theorem.
     Multiplication is computationally \emph{faster} than convolution, so this
     is the preferred method.
\end{frame}
%-------------------------------------------------------------------%
\begin{frame}{Temporal Fourier Transform}
    Need large pupils to image deeper focal planes,
    but this produces coma, primary astigmatism, and higher-order aberrations.
    $$(\mathbf{r},t) \rightarrow (\mathbf{r},\nu)$$
    $$ \check{H}_{+}(\mathbf{r},z,\nu) =
    \int\limits_{a<|\mathbf{r}-\mathbf{r}'|<b}\textrm{d}^{2}r'\check{G}_{+}
        (|\mathbf{r}-\mathbf{r}'|,z,\nu)\check{\psi}(\mathbf{r}',\nu) $$
\end{frame}
%-------------------------------------------------------------------%
\begin{frame}{Subjacent vs. Superjacent Vantage Holography}

    {\large\textcolor{cblue}{\textbf{Superjacent:}}}
    \begin{itemize}
        \item Wave propagated directly \emph{upward} from the source
            to surface
        \item applies to active regions.
    \end{itemize}
    {\large\textcolor{cblue}{\textbf{Subjacent:}}}
    \begin{itemize}
        \item inner radius, $a$, of the pupil annulus is much greater than
            the depth of the focal plane;
        \item applies to quiet sun (often the practical choice)
    \end{itemize}
\end{frame}
%-------------------------------------------------------------------%
\begin{frame}{Subjacent Vantage Holography}
    \begin{figure}
        \includegraphics[width=\textwidth]{fig_4.png}
    \end{figure}
\end{frame}
%-------------------------------------------------------------------%
\begin{frame}
    \begin{figure}
        \includegraphics[width=0.9\textwidth]{fig_3.png}
    \end{figure}
    \vspace{-0.25cm}
    \begin{itemize}
        \item $a$ = 15 Mm, $b$ = 45 Mm
        \item (a) \& (b): superjacent
        \item (c) mixed perspective
        \item (d), (e), \& (f): predominantly subjacent
    \end{itemize}
\end{frame}
%-------------------------------------------------------------------%
\begin{comment}
\begin{frame}{A word about diffraction limits}
    Waves with highest harmonic degree $l$ (lowest $\rho$)
    deliver the finest diffraction limit.
\end{frame}
\end{comment}
%-------------------------------------------------------------------%
\begin{frame}{Egression power maps}{Lindsey and Braun (1998)}
    \begin{columns}
        \column{0.5\textwidth}
        Top panels
        \begin{itemize}
            \item $\nu$ = 6 mHz ($\Delta\nu$ = 1 mHz)
            \item model annulus around ``Rorschach'' splotches
        \end{itemize}
        Bottom panels
        \begin{itemize}
            \item Contrasts between object and surroundings
        \end{itemize}
        \textcolor{cpink}{\textbf{All subjacent vantage!}}
        \column{0.6\textwidth}
        \begin{figure}
            \vspace{-1cm}
            \includegraphics[width=0.95\textwidth]{fig_5.png}
        \end{figure}
    \end{columns}
\end{frame}

\begin{comment}
\begin{frame}{Need more here!}
\end{frame}
\begin{frame}{Fundamental limitation}
    As sound speed increases with depth, wavelength
    \emph{increases}, which results in a coarser diffraction limit at
    any frequency.
\end{frame}
\end{comment}
%-------------------------------------------------------------------%
\begin{frame}{Acoustic Modeling Based on Holographic Images}
    Flexible procedures, such as inversions, would characterize the
    acoustic environment in pysical terms such as:
    \begin{itemize}
        \item acoustic emissivity
        \item acoustic opacity
        \item refractivity
        \item flow velocity
    \end{itemize}
    where the last two result from the application of phase-sensitive
    holography
\end{frame}
%-------------------------------------------------------------------%
\begin{comment}
\begin{frame}{Inversions (need more here)}
    Fredholm integral:
    $$ \langle|H_{+}(\mathbf{r},z)|^{2}\rangle =
    \int\textrm{d}^{2}\mathbf{r}'\int\textrm{d}z'g^{-1}
    (|\mathbf{r}-\mathbf{r}'|,z,z')S(\mathbf{r}',z') $$
    Source distribution:
    $$ S(\mathbf{r},z) = \int\textrm{d}^{2}\mathbf{r}'
        \int\textrm{d}z'g^{-1}(|\mathbf{r}-\mathbf{r}'|,z,z')
        \langle |H_{+}(\mathbf{r}',z')|^{2}  \rangle $$
\end{frame}
\end{comment}
%-------------------------------------------------------------------%
\begin{comment}
\begin{frame}{Phase-Sensitive Holography}
    Purpose is to incorporate the basic utilities of optical
    interferometry into solar interior diagnostics.
    The need for phase-sensitive holography is two-fold:
    \begin{enumerate}
        \item straight-forward quantitative probe of refractive anomalies
            that we expect from thermal perturbations
        \item Something else
    \end{enumerate}
    ``standard acoustic-power holography'' depends on anomalous sources
    and absorbers
\end{frame}
\end{comment}
%--------------------------------------------------------------------------%
\begin{frame}{\emph{gedanken} experiment}
    \vspace{-0.25cm}
    \begin{figure}
        \includegraphics[width=\textwidth]{fig_6.png}
    \end{figure}
    \vspace{-0.25cm}
    \begin{itemize}
        \item Produce waves that travel from right to left, into the
            refractive sample.
    \end{itemize}
\end{frame}
%--------------------------------------------------------------------------%
\begin{frame}{\emph{gedanken} experiment}
    \begin{itemize}
        \item No phase-shift: $\Delta{n} = 0$
        \item Phase-shift:
            \begin{itemize}
                \item $\Delta{n} = \Delta{c_{s}}/c_{s} \rightarrow$
                    refractive perturbation
                \item $\Delta{t} \sim a\Delta{n}/c \rightarrow$
                    time delay
                \item $\Delta{\phi} \sim 2\pi\nu a\Delta{n}/c \rightarrow$
                    phase shift
            \end{itemize}
        where $a$ is the characteristic diameter of the sample.
    \end{itemize}

    To relate $\Delta\phi$ to $H_{+}$ and $H_{-}$, define the
    \emph{temporal} Fourier transforms
    \begin{itemize}
        \item $ \check{H}_{+}(\mathbf{r},z,\nu)  $
            $ \leftrightarrow$
            $ {H}_{+}(\mathbf{r},z,t)  $
        \item $ \check{H}_{-}(\mathbf{r},z,\nu)  $
            $ \leftrightarrow$
            $ {H}_{-}(\mathbf{r},z,t)$.
    \end{itemize}
    Then
    $$ \Delta\phi = \arg\left(\left\langle
        \check{H}_{+}(\mathbf{r},z,\nu)
        \check{H}_{-}^{*}(\mathbf{r},z,\nu)
        \right\rangle
        _{\Delta\nu}
        \right) $$
\end{frame}
%--------------------------------------------------------------------------%
\begin{comment}
\begin{frame}{Correlations}
    Temporal correlation: $$
    C(\mathbf{r},z,\tau) = \int\textrm{d}t'H_{-}(z,\mathbf{r},t')
    H_{+}(z,\mathbf{r},t'+\tau) $$
    Extension of correlation for measuring horizontal flows: $$
    C(\mathbf{r},z,\mathbf{s},\tau) =
    \int\textrm{d}t'H_{-}(z,\mathbf{r},t')
    H_{+}(z,\mathbf{r}+\mathbf{s},t'+\tau) $$
    $$ \Delta\mathbf{r} = \mathbf{v}\Delta{t} = \mathbf{v}a\Delta{n}/c $$
    $$ \mathbf{U}(\mathbf{r},z) = \left\langle\check{H}_{-}^{*}
    (\mathbf{r},z,\nu)\nabla\check{H}_{+}(\mathbf{r},z,\nu)\right\rangle
    _{\Delta\nu} $$
\end{frame}
\end{comment}
%--------------------------------------------------------------------------%
\begin{frame}{Green's Functions}
    $$ G_{\pm}(|\mathbf{r}-\mathbf{r}'|,z,t-t') $$
    Characterizes the propagation of acoustic disturbances
    between a surface point,
    $(\mathbf{r}',0,t')$ and source point, $(\mathbf{r},z,t)$

    These propagations take place in the solar \emph{model}
    to which helioseismic observations, $\psi(\mathbf{r}',t')$.  are applied.
\end{frame}

\begin{frame}{Dispersionless acoustics}
    \begin{itemize}
        \item Pulse propagates in the form of an infintely thin \emph{wavefront}.
        \item $\mathbf{r}'$ responds with ripple characterized
            by the same infinitely sharp temporal profile as the source,
            but properly attenuated.
        \item The Green's function is invarient with
            respect to both time and horizontal translation.
    \end{itemize}
    $$ G_{+}(|\mathbf{r}-\mathbf{r}'|,z,t-t') =
    \delta\left(t-t'-T\left(|\mathbf{r}-\mathbf{r}'|,z\right)\right)
       f\left(|\mathbf{r}-\mathbf{r}'|,z\right)$$
       \begin{itemize}
           \item $T$: travel time
           \item $f$: amplitude of pulse
       \end{itemize}
\end{frame}

\begin{comment}
\begin{frame}{Dispersionless acoustics}
    $$ T(|\mathbf{r}-\mathbf{r}') = \int\limits_{{\Gamma}}\frac{\textrm{d}s}{c} $$
    $$ f^{2} = \frac{1}{4\pi{c}\cos\alpha}\frac{\sin\theta}{\sin\rho}
    \frac{\textrm{d}\theta}{\textrm{d}\rho} $$

    Acoustic energy flux density: $cf^{2}$
\end{frame}

\begin{frame}
    \begin{figure}
        \includegraphics[width=\textwidth]{fig_7.png}
    \end{figure}
\end{frame}
\end{comment}

%Regions of strong magnetic field ($B \sim ?$ G)
%\emph{reflect} \emph{p}-modes above the acoustic cutoff frequency
%(x mHz) back into the interior,
%while the \emph{quiet} sun ($B \sim 10$ G) \emph{absorbs} it.
\begin{frame}{Dispersionless acoustics}{Reflected waves}
    \begin{itemize}
        \item $\nu>5.5$ mHz absorbed by photosphere.
        \item $\nu<4.5$ mHz reflected from photosphere.\\
            Green's function is characterized by a sum of $n$ components, where
            each $n$ is a ``skip'', or reflection from the photosphere.
    \end{itemize}
    \begin{enumerate}
        \item subjacent: $\rho$ decreases as $\theta$ increases
        \item superjacent: $\rho$ increases after reaching a minimum
            as $\theta$ continues to increase toward 180$^{\circ}$.
    \end{enumerate}
\end{frame}

\begin{frame}{Single-skip holography}
    \begin{figure}
        \includegraphics[width=0.8\textwidth]{fig_8.png}
    \end{figure}
\end{frame}
\begin{frame}{Two-skip holography}
    \begin{figure}
        \includegraphics[width=0.7\textwidth]{fig_9.png}
    \end{figure}
\end{frame}

\begin{frame}{Dispersion}
    In reality, acoustic waves are \emph{significantly} dispersed
    near the photosphere.
\end{frame}

\begin{frame}
    \begin{figure}
        \includegraphics[width=0.8\textwidth]{fig_10.png}
    \end{figure}
\end{frame}

\begin{frame}
    \begin{figure}
        \includegraphics[width=\textwidth]{fig_11.png}
    \end{figure}
\end{frame}

\begin{frame}{Comparison to other techniques}
    \begin{itemize}
        \item time-distance helioseismology
            (aka.\ tomography)
        \item ring diagram analysis
    \end{itemize}
\end{frame}

\begin{frame}{Applications}
    \begin{itemize}
        \item acoustic glories
        \item acoustic moats and/or
        \item confirming $p$-mode absorption in sunspots (p. 267)
    \end{itemize}
\end{frame}

\begin{frame}{Movie!}
\end{frame}

\begin{frame}{Take-home points}
\end{frame}

%--------------------------------------------------------------%

\end{document}
