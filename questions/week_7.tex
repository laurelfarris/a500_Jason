\documentclass[11pt]{article}
\usepackage[margin=1in]{geometry}
\usepackage{fancyhdr}
\pagestyle{fancy}

\setlength{\parindent}{0em}
\setlength{\parskip}{0.5em}

\lhead{Laurel Farris}
\chead{ASTR 500 --- Week 7 Questions}
\rhead{15 April 2016}


\begin{document}

\begin{enumerate}
    \item \textbf{Explain why the 12 M$_{\odot}$ star loses mass according to
        Figure 1.}

        The mass falls sharply when the outer layers of the star
        become convective, and the star begins to shed mass.

    \item \textbf{Why would stellar cores increase their spin rate as
        they move into evolved stages?}

        As stars evolve off the main sequence, their outer envelope
        expands, while their cores contract.
        This would cause the core rotation rate to speed up as a result
        of conservation of angular momentum.

    \item \textbf{How much (by what factor) does Figure 5 predict
        that the spin rates of the inner core would increase from
        H burning to C burning?}

        The period drops from $\sim 10^{5}$ seconds for H-burning
        to $\sim 10$ seconds for C-burning. This gives an increase
        in the spin rate by a factor of around $10^{4}$.

    \item \textbf{From the same figure, by what factor do gravity
        waves halt such spinning up?}

        Gravity waves halt this by a factor of $10^{2}$.

\end{enumerate}

\end{document}
