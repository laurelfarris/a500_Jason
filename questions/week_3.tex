\documentclass[11pt]{article}
\usepackage[margin=1in]{geometry}
\pagenumbering{gobble}
\usepackage{fancyhdr}
\pagestyle{fancy}

\setlength{\parindent}{0em}
\setlength{\parskip}{0.5em}

\lhead{Laurel Farris}
\chead{ASTR 500 \- Week 3 Questions}
\rhead{04 March 2016}

\begin{document}

\begin{enumerate}
    \item \textbf{What has occurred to change the solar abundances
        of metals that these authors consider?
        I.e., what has been done recently by other
        authors to provide new abundances (less heavy elements) for the Sun.}

        The lower abundances of lighter metals were derived when other
        authors started analyzing three-dimensional models,
        and started to consider hydrodynamic effects and
        ``uncertainties in the atomic data and the observed spectra.''

    \item \textbf{How accurate were the ``old'' models compared to helioseismology
        of the sound speed in the interior before these new abundances were
        introduced?}

        Before the new models with different abundances of heavier
        elements were introduced, the radial profiles of sound
        speed and density for the models agreed well with helioseismological
        observations.

    \item \textbf{What is the reason, or point, of the BP+xx\% models,
        where xx is some number? What are the authors trying to do
        with such models?}

        The BP+21\% model was included to address the ``convection zone
        problem'': If the tabulated OPAL opacity near the tachocline
        (base of convective envelope) was increased by 21\%, the new models
        would reproduce the measured depth of the convective zone.

    \item \textbf{What is the fuss over the ``abundance problem''? (do you think)}

        Study of the sun helps us to understand several things, such as
        the origin and evolution of our solar system (specifically the
        material from which it formed), properties of other stars,
        and even stellar populations of other galaxies.
        Since so much of our understanding of the universe is based
        on what we know about the sun, it is important to be as
        accurate as possible when deriving its properties.

\end{enumerate}
\end{document}
