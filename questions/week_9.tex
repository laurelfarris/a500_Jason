\documentclass[11pt]{article}
\usepackage[margin=1in]{geometry}
\usepackage{amsmath}
\usepackage{fancyhdr}
\pagestyle{fancy}
\pagenumbering{gobble}
\setlength{\parindent}{0em}
\setlength{\parskip}{0.5em}

\lhead{Laurel Farris}
\chead{ASTR 500 --- Week 9 Questions}
\rhead{29 April 2016}


\begin{document}

\begin{enumerate}
    \item \textbf{Equations 1.14-1.15 are very important for this
    topic. First, show that by carrying out the procedure in the
    paragraph under Eq. 1.15, considering the delta function
    correlation, that you arrive at the power proportionality stated
    in the paragraph. Then describe how Fig. 1.8 accurately reflects
    this scenario. You need only describe the simple mathematics with
    words $\ldots$ no need to write full equations. But be specific.}
    
    The Fourier transform of a delta function (in general) is
    proportional to the integral of the product of the delta function
    and an exponential. After integrating, $P(k)$ is equal to the
    exponential since the integration of the delta function (the
    correlation function) is 1. Since the correlation function only
    has one value at $r_{*}$, this corresponds to a single frequency
    in Fourier space, as shown in figure 1.8.

    \item \textbf{Now, consider gamma=2 in Eq. 1.14. Using Eq. 1.15,
    show in this case that one should expect curves as plotted in Fig
    1.9. Again, you need only describe the simple mathematics with
    words $\ldots$ no need to write full equations. But be specific.}

    For $\gamma = 2$, the $r$ factors in the integrand cancel, and
    $P(k) = -\left(\frac{r_{o}}{ik}\right)e^{-ikr} \propto e^{-ikr}$.
    Thus, as $k$ increases, $P(k)$ should
    damp out, as shown in figure 1.9.

    \item \textbf{What is the era of recombination? How does Fig. 1.10
    depict this important era?}

    The era of recombination (more like the era of combination)
    started at the time after the big bang when the universe had
    cooled to a temperature low enough for nuclei and electrons
    to combine and form atoms, around $z \approx 1000$.
    This is illustrated in figure 1.10, where
    the mass profile for baryons ``stalls'' between the panel at
    $z = 1440$ and the one at $z = 848$. This overdensity for the
    baryons occurred because the combination of atoms removed the
    pressure on the baryons, but the pressure on photons was unaffected,
    allowing them to propagate away in the form of the CMB.



\end{enumerate}
\end{document}
