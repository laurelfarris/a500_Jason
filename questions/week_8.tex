\documentclass[11pt]{article}
\usepackage[margin=1in]{geometry}
\usepackage{fancyhdr}
\pagestyle{fancy}

\setlength{\parindent}{0em}
\setlength{\parskip}{0.5em}

\lhead{Laurel Farris}
\chead{ASTR 500 \- Week 8 Questions}
\rhead{22 April 2016}


\begin{document}

\begin{enumerate}
    \item \textbf{Why is rapid rotation important for generating
        the gravitational wave signal of a supernova?}

        Sometimes it just does.

    \item \textbf{Based on the amplitude of the fluid motions in the
        proto-neutron star (see figure 4) and their frequencies,
        construct a rough estimate of the kinetic energy contained
        in the oscillations of the proto-neutron star
        (within the inner 10 km) just after core bounce.
        How does this compare to the total explosion energy of a
        typical supernova (roughly $10^{51}$ erg)?}

        $$ E = \frac{1}{2}mv^2 = \frac{1}{2}(8M_{\odot})
        {\left( \frac{2\pi f}{k} \right)}^2
        \approx 3\times10^{62} \textrm{erg}$$

        This is higher by 11 orders of magnitude$\ldots$


    \item \textbf{Describe the source of this kinetic energy and where
        the energy goes as the oscillations damp out.}

        Assuming my calculation was correct (estimating the core mass as the
        total stellar mass), a lot of energy would have to be lost in order
        for the final explosion to have so much less. These GWs are able to
        propagate outwards and shock, dissipating their energy.
\end{enumerate}

\end{document}
