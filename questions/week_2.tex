\documentclass[11pt]{article}
\usepackage[margin=1in]{geometry}
\usepackage{fancyhdr}
\pagestyle{fancy}

\setlength{\parindent}{0em}
\setlength{\parskip}{0.5em}

\lhead{Laurel Farris}
\chead{ASTR 500 - Week 2 Questions}
\rhead{26 February 2016}

\begin{document}

\begin{enumerate}
    \item \textbf{What are the differences between ``bending waves'' and
    ``density waves'' in the ring system?}

    Bending waves are produced by periodic vertical forces, and
    density waves are produced by periodic radial, or azimuthal
    forces, which cause variation in the surface density of the rings.
    The latter are the type of waves observed and analyzed in the paper.

    \item \textbf{What evidence do the authors have that these waves are
    driven by internal modes on the planet, rather than from moons
    or something else?}

    The location of the C-ring features is not in an area
    that would result in resonance with any of the moons. The authors
    also state that satellite resonance would not produce the high
    speeds at which the oscillation patterns rotate around the planet.
    The observed speeds were more similar to those
    predicted from waves generated by oscillations interior to the
    planet itself.

    \item \textbf{What ``type'' of interior modes are the ones that are thought
    to produce the ring features?}

    The features were thought to be produced by ``low-order normal
    mode oscillations'', in other words, the fundamental  

    \item \textbf{What physical quantity are the authors measuring to show
    that there are some sort of waves in the particular ring area
    at which they are looking?}

    The raw data flux values from the stellar occultation
    were normalized to 1 (at a point where most of the starlight
    was being transmitted) to determine the transmission at every
    other location. These values were then converted to optical depth.
    Variations in optical depth reveal variation in density, though
    the authors wound up performing wavelet analysis on the raw flux
    data, not the optical depth.

    \item \textbf{What potential for understanding Saturn is opened up by
    these observations? I.e., can these waves be exploited for something?}

    The authors mention several ways that interior models of Saturn
    could be constrained using their data, such as the timescales over
    which interior oscillations are created and dissipated, and a
    possible correlation between the amplitude of the interior
    oscillations and the density variations in the resulting waves
    seen in the C-ring.


\end{enumerate}
\end{document}
