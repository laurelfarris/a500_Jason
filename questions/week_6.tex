\documentclass[11pt]{article}
\usepackage[margin=1in]{geometry}
\usepackage{fancyhdr}
\pagestyle{fancy}

\setlength{\parindent}{0em}
\setlength{\parskip}{0.5em}

\lhead{Laurel Farris}
\chead{ASTR 500 --- Week 6 Questions}
\rhead{01 April 2016}

\begin{document}

\begin{enumerate}
    \item \textbf{Why would more massive red giants be more susceptible to this
        magnetic trapping effect than lower-mass ones?}

    \item \textbf{Why don't the radial modes (l=0) get affected by internal
        magnetic fields?}

        The authors state that ``radial modes do not propagate within the inner core'',
        which is where magnetic fields trap the wave energy.

    \item \textbf{How does an oscillation behave in an ``evanescent'' region?}
        When waves reach the evanescent region, they are either reflected back
        toward the surface or ``tunnel through'' to the core, where their energy
        is effectively lost, and does not come back and contribute to the surface
        oscillations that we observe.

        The evanescent region is where the local lamb frequency becomes greater than
        the frequency of the wave travelling through the interior; it is the boundary
        between the radiative and convection zone. Part of the wave is reflected back
        toward the stellar surface, and part is transmitted through this zone into
        the radiative core, depending on the value of the mode degree, $l$.


    \item \textbf{Look at supplementary figure 1 (Figure S1). About how much of the
        star's mass is in the outer convection zone?}

        The convective envelope contains about 1.3 M$_{\odot}$, which
        is more than 80\% of the star's total mass (1.6 M$_{\odot}$).

\end{enumerate}
\end{document}
