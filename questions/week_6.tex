\documentclass[11pt]{article}
\usepackage[margin=1in]{geometry}
\usepackage{fancyhdr}
\pagestyle{fancy}

\setlength{\parindent}{0em}
\setlength{\parskip}{0.5em}

\lhead{Laurel Farris}
\chead{ASTR 500 --- Week 6 Questions}
\rhead{08 April 2016}

\begin{document}

\begin{enumerate}
    \item \textbf{Why would more massive red giants be more susceptible to this
        magnetic trapping effect than lower-mass ones?}

        A minimum magnetic field strength is required for oscillations to be
        suppressed ($B_{c,\min}$). According to figure 4(A),
        this value is higher for more massive stars, so they are less likely
        to be susceptible. However, $B_{c,\min}$ decreases as the stars
        evolve, so since massive stars evolve faster than lower mass stars,
        they'll reach a decreased value for $B_{c,\min}$ sooner.

    \item \textbf{Why don't the radial modes (l=0) get affected by internal
        magnetic fields?}

        The authors state that ``radial modes do not propagate within the inner core'',
        which is where magnetic fields trap the wave energy.

    \item \textbf{How does an oscillation behave in an ``evanescent'' region?}
        When waves reach the evanescent region, they are either reflected back
        toward the surface or ``tunnel through'' to the core, where their energy
        is effectively lost, and does not come back and contribute to the surface
        oscillations that we observe.

    \item \textbf{Look at supplementary figure 1 (Figure S1). About how much of the
        star's mass is in the outer convection zone?}

        The convective envelope contains about 1.3 M$_{\odot}$, which
        is more than 80\% of the star's total mass (1.6 M$_{\odot}$).

\end{enumerate}
\end{document}
