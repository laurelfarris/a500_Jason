\documentclass[11pt]{article}
\usepackage[margin=1in]{geometry}
\usepackage{fancyhdr}
\pagestyle{fancy}

\setlength{\parindent}{0em}
\setlength{\parskip}{0.5em}

\lhead{Laurel Farris}
\chead{ASTR 500 \- Week 5 Questions}
\rhead{01 April 2016}

% Copy this before editing!!!

\begin{document}

\begin{enumerate}
    \item \textbf{How are the luminosities obtained such that the target
        stars can be plotted on a real H-R diagram like in Figure 1?}

        Hipparcos data was available for the target stars, from which
        the distance can be calculated from parallax. If the distance
        to an object is known, then the apparent brightness can be
        combined with the distance to give intrinsic brightness, or
        luminosity.


    \item \textbf{What motivates the authors' statement ``$\ldots$for the
        red giants in our sample, interstellar reddening cannot be
        neglected'', while for the main-sequence stars it can be ignored?}

        The red giants are at a greater distance than the
        main-sequence stars, and therefore there will be more
        reddenening due to the presence of more dust between us and
        the stars.

    \item \textbf{3. Discuss a couple reasons why the radius comparison
        (seismo vs.\ interferometry) for giants does not seem to be in
        as good agreement as the comparison for dwarfs.}

        The authors mention that there are larger uncertainties in
        the asterseismology observations for the giant stars.
        The scaling relations (equations (3) and (4)) depend on mass,
        frequency separation, and temperature. It may be easier to
        detect dwarf oscillations, or to determine their masses, since
        evolved stars tend to the same mass.


\end{enumerate}
\end{document}
